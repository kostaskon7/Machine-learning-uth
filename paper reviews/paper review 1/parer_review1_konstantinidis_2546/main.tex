\documentclass{article}
\usepackage[utf8]{inputenc}
\title{Paper review:\\Business-driven data analytics: A conceptual modeling framework by Soroosh Nalchigar,Eric Yu}

\author{Konstantinos Konstantinidis AEM:2546}
\date{March 2021}

\begin{document}

\maketitle

\section{Introduction}
\subsection{What is the paper's main focus?}
The paper focuses on an effective way of using data analytics in a business environment by applying the conceptual modeling framework that the authors suggest.The difficulty derives mostly from combining the strategies and decisions of a company with implementing analytical systems, like databases or machine learning methods. The rapid advance of the machine learning domain, as well as in platforms for handling large datasets add to this challenge .Conceptual modeling in scientific research is focused on defining suitable forms of higher abstraction.The efficient development of information systems very crucial and that is what the framework is trying to achieve. 
\subsection{How do the authors approach the problems?}
The authors approach the framework through three modeling views, the business, the analytics design and the data preparation view.The goal of the business view's is to try to export the useful information out of the data so they can predict future values of the analysis ,it also intends to classify the data for better analysis and prescribe the best alternative solutions. The Analytics Design View describes the design of the analytical goals,the machine learning methods,the softgoals and performance review.The Data Preparation View is the processing of mechanisms and data preparation for the developers to be able to design algorithms with a better understanding of the task at hand.

\section{Main points}
\subsection{How metamodels work?}
Following section 1.2, the three modeling views can be analyzed deeper in order to find what constitutes them.The Business view combines the goals of the questions and the decisions that the business is trying to achieve.In order to do that,the model uses data insights which are patterns derived from data analysis and data prediction methods like machine learning.\\
The Analytics Design View utilitates the processing of data by combining our current knowledge about algorithms and how they perform with different performance indicators and goals that we are trying to achieve.\\
The Data Preparation View is is divided into many Tasks,some of which are data cleaning (only the useful are processed),data reduction(smaller size for time and power efficiency),data integration(merging from different sources).The preparation is a really crucial step of the analysis because all the modeling framework parts depend on it.
\subsection{How Cataloguing analytics work?}
The Cataloguing analytics area is a valuable part of the modeling,it is used to arrange and demonstrate the understanding of  the conceptual models. They provide generic designs and solutions for repeatable data analysis problems. The Business Questions Catalogue is used to attribute knowledge to common business questions and associate them with relevant analytics goals.Usual questions are which client will stop using the business product or when the product is out of date.The questions' nature may vary from past to present events in order to find where mistakes where made or what to do moving forward.\newline
The Algorithms Catalogue is utilized to address machine learning algorithms with different types of analysis goals.Customary cases that the catalogue is used are clustering algorithms for data organization and finding the most efficient one. The Data Preparation Techniques Catalogue, as the name suggests, tries to take better use of the data for the analysis afterwards.Common uses are data normalization and transformation.
\section{Conclusion}
\subsection{Paper results}
After testing,the researchers came to the conclusion that the conceptual framework is applicable in search of analysis solutions.The practical testing's downside of the conceptual model was that it was experimented only with the same analytics team.On the other hand,it only showed positive results for the business efficiency in the data analysis department.
The paper provides a theoretical view of the Business modeling for data analytics and aims to help business growth and better data analysis usage. 

\end{document}
